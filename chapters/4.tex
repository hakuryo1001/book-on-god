\chapter{有人話真相係}

有人話真相係就係點樣樣嘅嘢。
% 1. **Correspondence Theory**:  
%    - **Truth is what actually is.**
有人話真相係同你信開嗰套講得埋一齊嘅嘢。
% 2. **Coherence Theory**:  
%    - **Truth is what coheres with a set of beliefs.**
有人話真相係實際上行得通、有用嘅嘢。
% 3. **Pragmatic Theory**:  
%    - **Truth is what works or is useful in practice.**
有人話真相係齋方便噏野嘅嘢,冇乜特別意思。
% 4. **Deflationary Theory**:  
%    - **Truth is merely a linguistic convenience.**
有人話真相係大家坐埋枱傾吓傾吓,噏吓噏吓、唔經唔覺約定俗成揼出嚟嘅嘢。
% 5. **Constructivist Theory**:  
%    - **Truth is constructed through social processes and interactions.**
有人話真相係嗰句講出嚟同現實啱牙相符嘅嘢。
% 6. **Semantic Theory of Truth**:  
%    - **Truth is defined by the correspondence between language and reality.**
有人話真相係根本就係多餘嘅嘢諗,斷宣一句嘢講係堅嘅根本就冇添加到任何新嘅資訊。
% 7. **Redundancy Theory**:  
%    - **Truth is redundant; asserting a statement is true adds no new information.**
有人話真相係用靠得住嘅方法咇出到出嚟嘅嘢。
% 8. **Reliabilist Theory**:  
%    - **Truth is what is produced by a reliable method of belief formation.**
有人話真相係有好多個好多種嘅,同埋係按情況而變嘅,喺唔同嘅領域度會唔同。
% 9. **Pluralist Theory**:  
%    - **Truth is multiple and context-dependent, varying across different domains.**
有人話真相係無論你信唔信,都照樣存在嘅嘢。
% 10. **Alethic Realism**:  
%    - **Truth is an objective feature of reality that exists independently of our beliefs.**


% 有 人 話 野靚嘅呢個特質 係 嘅 一 󱀩野。
有 人 話 野靚嘅呢個特質 係 有和諧、有比例、睇落舒服嘅 嘢 所 呈現 嘅 一 󱀩野。
% 1. **Classical Theory**  
%    - **Philosophers**: Plato, Aristotle  
%    - **Beauty is what embodies harmony and proportion.**
有 人 話 野靚嘅呢個特質 係 會令人 動人心弦 撩起人心 嘅 嘢 所 呈現 嘅 一 󱀩野。
% 2. **Romantic Theory**  
%    - **Philosophers**: Immanuel Kant, Friedrich Schiller  
%    - **Beauty is what inspires deep emotional resonance.**
有 人 話 野靚嘅呢個特質 係 會令人 睇、聽、感受到之後 覺得好享受 嘅 經驗 所 呈現 嘅 一 󱀩野。
% 3. **Aesthetic Experience Theory**  
%    - **Philosophers**: John Dewey, Susanne Langer  
%    - **Beauty is what arises from a rich aesthetic experience.**
有 人 話 野靚嘅呢個特質 係 賞心悅目,畀人 五官享樂 嘅 嘢 所 呈現 嘅 特質,而且係一樣 主觀 受授予 嘅 一 󱀩野。
% 4. **Subjective Theory**  
%    - **Philosophers**: David Hume, George Santayana  
%    - **Beauty is what delights the senses and is subjectively perceived.**


有 人 話 野靚嘅呢個特質 係 靚嘢本身就有嘅 一 󱀩味道,見到就會覺得正,唔關你點諗事。
% 5. **Objective Theory**  
%    - **Philosophers**: Leo Tolstoy, Clive Bell  
%    - **Beauty is an inherent quality in objects that elicits appreciation.**

   6. Formalism
有 人 話 野靚嘅呢個特質 係 喺 藝術品 嘅 形式、度 一 󱀩野


% 6. **Formalism**  
%    - **Philosophers**: Clement Greenberg, Roger Fry  
%    - **Beauty is found in the formal qualities and structure of an artwork.**

   7. Cultural Relativism

有人話靚唔靚要睇地方同文化,邊度流行咩就咩靚,冇絕對標準。

8. Evolutionary Theory

有人話靚就係睇落健康、有基因優勢、啱繁殖嗰啲,見到就想交配嗰隻 feel。

9. Postmodern Theory

有人話靚就係挑機、反傳統、打破規矩嘅嘢,主觀到仆街,邊個都可以話係靚。

10. Utilitarian Theory

有人話靚就係令多人爽、少人痛,淨係睇好唔好受,唔講其他。

7. **Cultural Relativism**  
   - **Philosophers**: Edward Said, Richard Rorty  
   - **Beauty is what is defined by cultural norms and context.**

8. **Evolutionary Theory**  
   - **Philosophers**: Charles Darwin, Geoffrey Miller  
   - **Beauty is what signals health and reproductive fitness.**

9. **Postmodern Theory**  
   - **Philosophers**: Jean-François Lyotard, Michel Foucault  
   - **Beauty is what challenges established norms and is subjective.**

10. **Utilitarian Theory**  
    - **Philosophers**: John Stuart Mill, Jeremy Bentham  
    - **Beauty is what maximizes pleasure and minimizes pain.**



Here are ten theories of the "good" along with their representative philosophers and summaries:

1. **Utilitarianism**  
   - **Philosophers**: Jeremy Bentham, John Stuart Mill  
   - **Good is what maximizes pleasure and minimizes pain.**

2. **Deontological Ethics**  
   - **Philosophers**: Immanuel Kant  
   - **Good is what aligns with moral duties and universal laws.**

3. **Virtue Ethics**  
   - **Philosophers**: Aristotle, Alasdair MacIntyre  
   - **Good is what promotes human flourishing and virtuous character.**

4. **Ethical Relativism**  
   - **Philosophers**: Ruth Benedict, Franz Boas  
   - **Good is what is defined by cultural norms and societal context.**

5. **Divine Command Theory**  
   - **Philosophers**: St. Augustine, William of Ockham  
   - **Good is what is commanded by a divine being.**

6. **Natural Law Theory**  
   - **Philosophers**: Thomas Aquinas, John Finnis  
   - **Good is what is in accordance with human nature and reason.**

7. **Hedonism**  
   - **Philosophers**: Epicurus, Jeremy Bentham  
   - **Good is what brings the greatest pleasure to the individual.**

8. **Social Contract Theory**  
   - **Philosophers**: Thomas Hobbes, John Locke, Jean-Jacques Rousseau  
   - **Good is what is agreed upon for the benefit of society.**

9. **Pragmatic Ethics**  
   - **Philosophers**: John Dewey, William James  
   - **Good is what proves effective in solving problems and improving lives.**

10. **Care Ethics**  
    - **Philosophers**: Carol Gilligan, Nel Noddings  
    - **Good is what nurtures relationships and promotes care for others.**


Here are ten theories of "the right" along with their representative philosophers and summaries:

1. **Utilitarianism**  
   - **Philosophers**: Jeremy Bentham, John Stuart Mill  
   - **The right is what maximizes pleasure and minimizes pain for the greatest number.**

2. **Deontological Ethics**  
   - **Philosophers**: Immanuel Kant  
   - **The right is what conforms to moral duties and categorical imperatives.**

3. **Virtue Ethics**  
   - **Philosophers**: Aristotle, Alasdair MacIntyre  
   - **The right is what cultivates virtuous character and promotes human flourishing.**

4. **Social Contract Theory**  
   - **Philosophers**: Thomas Hobbes, John Locke, Jean-Jacques Rousseau  
   - **The right is what is agreed upon for the benefit of society and its members.**

5. **Divine Command Theory**  
   - **Philosophers**: St. Augustine, William of Ockham  
   - **The right is what is commanded by God or aligns with divine will.**

6. **Consequentialism**  
   - **Philosophers**: Peter Singer, R. M. Hare  
   - **The right is determined by the outcomes and consequences of actions.**

7. **Rights Theory**  
   - **Philosophers**: John Locke, Robert Nozick  
   - **The right is what respects and upholds individual rights and freedoms.**

8. **Care Ethics**  
   - **Philosophers**: Carol Gilligan, Nel Noddings  
   - **The right is what fosters care and responsibility in relationships.**

9. **Pragmatic Ethics**  
   - **Philosophers**: John Dewey, William James  
   - **The right is what effectively addresses problems and promotes well-being.**

10. **Moral Intuitionism**  
    - **Philosophers**: G. E. Moore, W. D. Ross  
    - **The right is what is intuitively recognized as morally correct.**
