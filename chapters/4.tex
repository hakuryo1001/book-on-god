我越嚟越唔明,點解會有人真係由心信無神論。無神論唔單止係一個哲學立場,佢本身係一個極其恐怖嘅世界觀。佢所謂嘅「解放」,其實係一種將人由秩序、由意義、由善惡觀裏面抽離出嚟嘅暴力行為,就好似「解放」上海咁——表面上係自由,實際上係摧毀。

試諗下,喺無神論嘅框架之下,所有存喺樹都變成偶然,善惡冇根據,美醜冇標準,正義只係權力平衡嘅幻覺。人嘅一切追求,無論係愛、理想、犧牲、信念,最終都會被仲原成化學反應同基因算法。咁樣嘅宇宙,係凍冰冰、無情緒、無目嘅、無中心嘅。喺呢個意義上,無神論唔止係一個思想,而係一個地獄嘅形態——你仍然生存,但冇任何理由、冇任何歸宿。

無論你點去定義「神」——無論係創造者、終極之善、終極之美、終極之義——否定任何一樣,都等於否定咗你自己存喺樹嘅根本意義。因為一個能夠理解美、善同義嘅人,若然突然相信呢啲嘢全都係幻覺,佢嘅靈魂應該會抽筋,理智應該會崩潰。

所以我真係唔明,點樣可以情緒上面接受一個無神嘅世界。除非一個人已經完全麻木、唔再尋求意義,只剩下生理層面嘅慾望孖反射。否則,正常人如果真誠咁面對無神論,就應該會陷入深深嘅恐懼同絕望入面。

無神論表面上係解放,其實係剝奪。佢奪走咗人對自身以外嘅敬畏、對秩序嘅信任、對存喺樹嘅愛。佢令一切都變成功能,一切都可以被計算,一切都失去神聖。喺呢種狀態之下,人雖然仲行喺地面,但精神早已跌入地獄。
