\chapter{有人話真相係}

有人話真相係就係點樣樣嘅嘢。
% 1. **Correspondence Theory**:  
%    - **Truth is what actually is.**
有人話真相係同你信開嗰套講得埋一齊嘅嘢。
% 2. **Coherence Theory**:  
%    - **Truth is what coheres with a set of beliefs.**
有人話真相係實際上行得通、有用嘅嘢。
% 3. **Pragmatic Theory**:  
%    - **Truth is what works or is useful in practice.**
有人話真相係齋方便噏野嘅嘢,冇乜特別意思。
% 4. **Deflationary Theory**:  
%    - **Truth is merely a linguistic convenience.**
有人話真相係大家坐埋枱傾吓傾吓,噏吓噏吓、唔經唔覺約定俗成揼出嚟嘅嘢。
% 5. **Constructivist Theory**:  
%    - **Truth is constructed through social processes and interactions.**
有人話真相係嗰句講出嚟同現實啱牙相符嘅嘢。
% 6. **Semantic Theory of Truth**:  
%    - **Truth is defined by the correspondence between language and reality.**
有人話真相係根本就係多餘嘅嘢諗,斷宣一句嘢講係堅嘅根本就冇添加到任何新嘅資訊。
% 7. **Redundancy Theory**:  
%    - **Truth is redundant; asserting a statement is true adds no new information.**
有人話真相係用靠得住嘅方法咇出到出嚟嘅嘢。
% 8. **Reliabilist Theory**:  
%    - **Truth is what is produced by a reliable method of belief formation.**
有人話真相係有好多個好多種嘅,同埋係按情況而變嘅,喺唔同嘅領域度會唔同。
% 9. **Pluralist Theory**:  
%    - **Truth is multiple and context-dependent, varying across different domains.**
有人話真相係無論你信唔信,都照樣存在嘅嘢。
% 10. **Alethic Realism**:  
%    - **Truth is an objective feature of reality that exists independently of our beliefs.**


% 有 人 話 野靚嘅呢個特質 係 嘅 一 󱀩野。
有 人 話 野靚嘅呢個特質 係 有和諧、有比例、睇落舒服嘅 嘢 所 呈現 嘅 一 󱀩野。
% 1. **Classical Theory**  
%    - **Philosophers**: Plato, Aristotle  
%    - **Beauty is what embodies harmony and proportion.**
有 人 話 野靚嘅呢個特質 係 會令人 動人心弦 撩起人心 嘅 嘢 所 呈現 嘅 一 󱀩野。
% 2. **Romantic Theory**  
%    - **Philosophers**: Immanuel Kant, Friedrich Schiller  
%    - **Beauty is what inspires deep emotional resonance.**
有 人 話 野靚嘅呢個特質 係 會令人 睇、聽、感受到之後 覺得好享受 嘅 經驗 所 呈現 嘅 一 󱀩野。
% 3. **Aesthetic Experience Theory**  
%    - **Philosophers**: John Dewey, Susanne Langer  
%    - **Beauty is what arises from a rich aesthetic experience.**
有 人 話 野靚嘅呢個特質 係 賞心悅目,畀人 五官享樂 嘅 嘢 所 呈現 嘅 特質,而且係一樣 主觀 受授予 嘅 一 󱀩野。
% 4. **Subjective Theory**  
%    - **Philosophers**: David Hume, George Santayana  
%    - **Beauty is what delights the senses and is subjectively perceived.**


有 人 話 野靚嘅呢個特質 係 靚嘢本身就有嘅 一 󱀩味道,見到就會覺得正,唔關你點諗事。
% 5. **Objective Theory**  
%    - **Philosophers**: Leo Tolstoy, Clive Bell  
%    - **Beauty is an inherent quality in objects that elicits appreciation.**

   
有 人 話 野靚嘅呢個特質 係 喺 有形式 同 有\jlr{木}{}\jlr{木}{}嘅 野 所 呈現 嘅 一 󱀩野。
% 6. **Formalism**  
%    - **Philosophers**: Clement Greenberg, Roger Fry  
%    - **Beauty is found in the formal qualities and structure of an artwork.**

有 人 話 野靚嘅呢個特質 係 睇 文化習俗 同 場合 話事, 邊度流行咩就咩靚 嘅 一 󱀩野。


% 7. **Cultural Relativism**  
%    - **Philosophers**: Edward Said, Richard Rorty  
%    - **Beauty is what is defined by cultural norms and context.**



有 人 話 個好字 係 可以 將 歎嘅感覺 撐到最大 同 將 痛苦 削到最少 嘅 嘢。
% 1. **Utilitarianism**  
%    - **Philosophers**: Jeremy Bentham, John Stuart Mill  
%    - **Good is what maximizes pleasure and minimizes pain.**
有 人 話 個好字 係 同 孭義氣鑊 同埋 普世 規矩 啱牙對齊 嘅 嘢。
% 2. **Deontological Ethics**  
%    - **Philosophers**: Immanuel Kant  
%    - **Good is what aligns with moral duties and universal laws.**
有 人 話 個好字 係 推動埋晒 啲 人類 豐盛 同 高尚 情操 嘅 嘢。
% 3. **Virtue Ethics**  
%    - **Philosophers**: Aristotle, Alasdair MacIntyre  
%    - **Good is what promotes human flourishing and virtuous character.**

有 人 話 個好字 係 人定嘅 同 風俗噏出嚟嘅。
% 4. **Ethical Relativism**  
%    - **Philosophers**: Ruth Benedict, Franz Boas  
%    - **Good is what is defined by cultural norms and societal context.**

有 人 話 個好字 係 一個 有神性嘅 有人格體 話 係 好 所以就係好嘅嘢。

% 5. **Divine Command Theory**  
%    - **Philosophers**: St. Augustine, William of Ockham  
%    - **Good is what is commanded by a divine being.**

有 人 話 個好字 係 乎 人類 天性、自然,同理性黹官嘅嘢。
% 6. **Natural Law Theory**  
%    - **Philosophers**: Thomas Aquinas, John Finnis  
%    - **Good is what is in accordance with human nature and reason.**
有 人 話 個好字 係 好玩 好歎 嘅嘢。
% 7. **Hedonism**  
%    - **Philosophers**: Epicurus, Jeremy Bentham  
%    - **Good is what brings the greatest pleasure to the individual.**
有 人 話 個好字 係 被認同 係 可以 益到 社會 嘅 嘢。
% 8. **Social Contract Theory**  
%    - **Philosophers**: Thomas Hobbes, John Locke, Jean-Jacques Rousseau  
%    - **Good is what is agreed upon for the benefit of society.**

有 人 話 個好字 係 證明到 可以 改善到 同 解決到 社會 問題 嘅 嘢。
% 9. **Pragmatic Ethics**  
%    - **Philosophers**: John Dewey, William James  
%    - **Good is what proves effective in solving problems and improving lives.**

有 人 話 個好字 係 培育到 人同人之間嘅 挐褦 同 推進到 對他人嘅關懷 嘅 野。
% 10. **Care Ethics**  
%     - **Philosophers**: Carol Gilligan, Nel Noddings  
%     - **Good is what nurtures relationships and promotes care for others.**

有 人 話 個好字 係 問心 就知 係 乜 嘅 嘢。
    % 孟子



有 人 話 乜嘢係 就係 可以 令 最多人 開心 同 減少 最多人 痛苦 嘅 嘢。
% 1. **Utilitarianism**  
%    - **Philosophers**: Jeremy Bentham, John Stuart Mill  
%    - **The right is what maximizes pleasure and minimizes pain for the greatest number.**

有 人 話 乜嘢係 就係 符合 道德 責任 同 道德命令 嘅 嘢。
% 2. **Deontological Ethics**  
%    - **Philosophers**: Immanuel Kant  
%    - **The right is what conforms to moral duties and categorical imperatives.**
有 人 話 乜嘢係 就係 培育埋晒 德性 同埋 推動埋晒 人類 幸福 同 盛放 嘅嘢。
% 3. **Virtue Ethics**  
%    - **Philosophers**: Aristotle, Alasdair MacIntyre  
%    - **The right is what cultivates virtuous character and promotes human flourishing.**

有 人 話 乜嘢係 就係 大家 認同 係 會 益到 社會 及 其 成員 嘅 嘢。

% 4. **Social Contract Theory**  
%    - **Philosophers**: Thomas Hobbes, John Locke, Jean-Jacques Rousseau  
%    - **The right is what is agreed upon for the benefit of society and its members.**

有 人 話 乜嘢係 就係 被 神 命令 同埋 符合 神 嘅 意志 嘅 嘢。
% 5. **Divine Command Theory**  
%    - **Philosophers**: St. Augustine, William of Ockham  
%    - **The right is what is commanded by God or aligns with divine will.**
有 人 話 乜嘢係 就係 睇 後果 同 孻屘點樣 所決定 嘅 嘢。
% 6. **Consequentialism**  
%    - **Philosophers**: Peter Singer, R. M. Hare  
%    - **The right is determined by the outcomes and consequences of actions.**
有 人 話 乜嘢係 就係 尊重 同 維護 個人 權利 同 自由 嘅 嘢。

有 人 話 乜嘢係 就係 嘅嘢,就算 一個 做緊 佢 嘅 人 都 冇; 乜嘢係唔 就係 唔嘅嘢,就算 人人 都 做緊。


% 7. **Rights Theory**  
%    - **Philosophers**: John Locke, Robert Nozick  
%    - **The right is what respects and upholds individual rights and freedoms.**

ee