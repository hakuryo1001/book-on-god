\chapter{冇 㕦 呢句野 點話}
講 世間上 有冇 㕦 呢個問題,講 證明 嘅話,我 就 想問,唔知 可唔可以 搵到一個係話㕦唔存在嘅建設性證明呢吓?咩叫建設性嘅證明呢?姐係個證明,個,唔用歸謬法嘅,唔係詏到孻屘就話有矛盾所以之前講過嘅詏落嘅全部都站唔住腳,唔係講到大尾篤見到唔對路所以就可以好有信心話開初講嘅假設係流嘅,所以個假設嘅倒返轉先至係堅嘅。咁樣嘅詏論就係建設性喇。我嘅問題就係,要證明㕦唔存在,可唔可以搵到一個建設性嘅呢?

認真諗下諗下,我都幾肯定應該係做唔到。事關本身個要求好似就有少少矛盾,有少少又要又唔要。點解咁話呢?嗱,唔要歸謬呀嘛,咁姐係你想砌野出嚟啦。但事關你點樣砌一個講法去話一樣野唔存在呢?如果你係想你個詏論係建設性嘅,你就姐係係砌緊啲野出嚟。砌嘅過程就係立論同證明嘅過程佢自己本身。你要砌,你就要講嗰舊野係點點點。咁姐係你係瞄住果舊野去出發。你嘅目的就係透過你嘅砌積木式嘅立論去講,嗱,因為我地知道呢舊野同呢舊野都存在,所以將佢哋咁樣咁樣組合埋嘅嘢呢都存在。存在住嘅呢個特質 透過 你 砌嘅呢個行為 由 一個 嘢講 保存到 去 另外 一個嘢講。咁試問,你點樣可以去到孻屘係飆到「唔存在」出嚟呢?你由乜乜乜點點點度開始,砌咗呢舊野,跟住又切到呢舊嘢,跟住呢舊野,一舊一舊存在生\tone{牛}{1}\tone{牛}{1}躉晒喺度,係出唔到舊冇野——唔啱,可能話「唔係野」先至啱——係出唔到出嚟架啵。你重要唔係要證明有啲咩「冇野」或者「唔係野」——如果呢幾舊野到嘅話——我地要證明嘅係一舊叫㕦嘅嘢唔存在。講到尾,要試問嘅就係諸君我等何以茲叵能之物理建黹立之乎?

咁姐係話,如果你唔要係講到淆底先至話冇 㕦嘅話,你有冇辦法直接講得出點解㕦唔存在呢?

歸謬法,其實運作上 係 同構乎 一方 俾人 質問,畀人\lr{言}{},卒之畀人\lr{言}{}冇嘢講 講唔同喇,要淆底,推翻晒之前講嘅嘢。反問呢個手段,「乜可以咩」,本質上就係要搭嘅嗰個話唔可以。話唔可以就係承認淆底嘅嗰一刻,就係歸謬嘅一刻,就係子之矛攻子之盾一\lr{声}{󱐆}嘅嗰一剎那。撞板,就係我地冇得唔淆底嘅嗰一刻。淆底,就係我地打倒晒之前最開初嗰句嘢講,並且要樹立嗰句嘢講調返轉嘅命令。

歸謬法性嘅證明呢,係本質上同建設性嘅證明有唔同㗎所以。依度順帶一提,廣東話嘅日常市井立論習慣,係好撚多歸謬法性嘅嘢講同手段。而既然歸謬本身因為係淆底,就必然要嗰歸謬嘅一方吞左啖氣佢去歸謬,個行為係有認低威嘅成份嘅。要認低威,就要放低,咁普通人就自然\lr{}{色}\lr{}{色}󱛒󰸒。個々都\lr{}{色}々\lr{}{色}々,咪容易鬧到面紅耳赤火遮眼,跟住你又唔服我我又唔服你囉,最後未都係講唔到個真理出嚟。所以呢,講廣東話嘅人,好似零舍臭脾氣就係有部份係喺依度嚟。

言歸正傳,講咗咁耐拗轉晒條腰都係講到𦧲晒脷,斷唔會要話㕦唔存在係可以用一個建設性嘅證明可以做得到。敢亦姐係話呢,斷任何一個話乜乜乜係唔可能嘅嘢講,要講得通嘅話,就一定要係裝埋牆,裝版,畀矛盾叫我地掉頭走,我地先至可以講得通話乜乜乜係冇可能。

但只不過呢,我地要記住,我地之所以可以撞\tone{到}{2}版撞\tone{到}{2}牆,係因為個迷宮有捧牆喺度。個迷宮係點,係取決於個迷宮嘅設計。個迷宮個\ruby{}{design}之所以係咁嘅樣而唔係另外一個樣,係因為我地揀咗呢個迷宮嚟行。

個比喻可能扯得太遠,要講得具體返啲。我地之所以會撞板,遇到矛盾,係因為我地用嘅「公理」,姐係,係注定硬會導致我地行到呢個位就會遇到個矛盾——因為我地開初嘅嗰個想證明嘅嗰個嘢講嘅調返轉,係同乍公理係一定唔啱牙,水溝油,有矛盾。而呢啲嘅公理,係我地服嘅開波嘢講,我地服嘅 底層嘢講 ——姐係我地講到底我地󰇞我嘅野,我地當係堅係真,唔使問嘅嘢。我地當係堅係真,係因為我地信呢啲嘢講。我地服呢野嘢講。因為我地覺得佢哋係所謂嘅不言自明,自證然者,唔使問亞季之事物。

唔使問姐,係唔係姐係唔問得呢?實問得㗎,但係會好難搞咁解。但係我地即管試吓啦吓。我地點解信呢所謂嘅公理呢吓?其實你可以唔信唔\lr{貝}{󰇞}唔服。咁你唔信唔\lr{貝}{󰇞}唔服,我咪試吓講啲理由過你,講下啲蠽到你直覺嘅嘢講囉。嗱,咁你聽完,你可能覺得,啊,到聽落講得通啊,信\lr{貝}{󰇞}服喇,咁咪收工囉。但係如果你照舊唔信唔\lr{貝}{󰇞}唔服嘅,咁我咪再試吓講多啲蠽得到你直覺嘅嘢囉。咁來來回回拉拉鋸鋸,你其實都係可以唔信唔\lr{貝}{󰇞}唔服架啵——你唔係痴線或者戇柒,你係的確可以真誠真心信唔到,\lr{貝}{󰇞}唔落,心唔服。唔過骨,係因為過唔到你骨。過唔到你骨,係因為你唔肯比佢過骨。你唔\lr{耳}{制}畀佢過骨,係因為你畀唔落。就好似一個要違反天地良心,仍然覺得有合理懷疑嘅陪審員,你唔肯唔\lr{耳}{制}。姐係,你揀咗啲嘢。你做咗選擇。

咁亦姐係意味住話,信\lr{貝}{󰇞}服一條所謂嘅公理,其實只不過係揀咗條公理嚟信\lr{貝}{󰇞}服咁解。

而你信\lr{貝}{󰇞}服得一條嘢嚟升佢上神台,就姐係你揀咗佢嚟信\lr{貝}{󰇞}服。

正如先頭所講,你信\lr{貝}{󰇞}服埋晒做公理嘅嘢,決定住你去信\lr{貝}{󰇞}服埋晒啲其他嘅嘢,或者你斷估話埋晒嘅其他嘢講,會唔會同你信\lr{貝}{󰇞}服嘅公理矛盾撞板,攪到你拗到某個為嘅時候就要淆底反口。

話一樣嘢係唔可能,淨係可以靠歸謬法先至講得通。以之因為歸謬法本質上係建基於你信服一啲特定嘅公理。你要首先認咗嗰啲公理,先至可以揸住啲公理當牌揸正牌咁講話:「根據我信\lr{貝}{󰇞}服嘅呢啲公理,乜乜乜係一定唔通、一定唔成立、一定唔可能。」

咁即係話,其實係有可能,你信\lr{貝}{󰇞}一啲嘢做公理嘅嘢講,佢哋所可以孳乳到出嚟嘅矛盾,同另外一乍等住比你信\lr{貝}{󰇞}服做公理嘅嘢講,所可以孳乳到出嚟嘅矛盾,可能會係唔一樣。你信呢個就可能一定會係呢到撞板,但係信嗰個就可能你個矛盾從來到唔會出現,行到去嗰樹就已經消散左,等住你係一條康莊大道,比你一炮過講到尾。信呢個就會有呢個矛盾唔會有嗰個矛盾。同時間,你要有呢個呢個矛盾你就可能要信呢一組組嘅公理組合。咁,你要有啲咩矛盾,要透過歸謬法嚟講得到話乜乜乜係冇可能嘅話,你就要執藥咁執執執呢個嗰個公理出嚟。

咁再檻多一步,一󱝚公理同一󱝚矛盾係有互對嘅關係住,或者係所謂「雙射」嘅關係。一󱝚公理,係一對一對住一󱝚矛盾。要乜嘅公理,就有乜嘢嘅矛盾。公理定矛盾,矛盾定公理。要乜嘢矛盾,就有乜嘢公理。乜乜乜公理係若且僅若啲乜乜乜矛盾。再講得中啲,就係公理就係矛盾。堅野同流野係互定其定義嘅。
