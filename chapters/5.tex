\chapter{理性󱝚邊界}

野係理性?理性係由頭落到手指尾啄啄一個來佬貨。講廣東話嘅人係可以話自古以來都冇理性嘅。廣東話裏面可以用嚟討論理性嘅資源共㗎撐 根本就淨係得雞碎咁多;要搵啲講理性係啲乜東東嘅文啊書啊根本就係苛求太監論床事。

泰西鬼佬講嘅理性,用我地廣東話要搵返個咁上下嘅講法呢,可能就係「諗通講通嘢嘅」



There is a boundary to rationality. What lies beyond there? it is wittgenstein’s whatever one cannot speak of one must remain silent. It is Jonah and the whale. It is 塞翁失馬 塞翁得馬。 it is isaiah chapter 55 verse 8 - my ways are not your ways. It is reflexivity. It is karma. It is serendipity. It is revelation. It is the integral factor rule. A computational competent man would compute his rational conclusions without fault. A wise man would know where his rationality ends and let God work his powers. 


理性有個界限。界限之外係乜嘢呢?就係維根斯坦講嗰句——「凡係講唔到嘅,就要保持沉默。」就係約拿喺鯨魚肚入面。就係「塞翁失馬,焉知非福」。就係《以賽亞書》第五十五章第八節:「我嘅道路唔係你哋嘅道路。」就係反身性。就係因果報應。就係機緣巧合。就係啟示。就係積分因子法則。一個計算能力完全正確嘅人,可以冇錯咁算到佢嘅理性結論。但一個智慧嘅人,會知道自己嘅理性去到邊度就完,然後交畀上帝去發揮祂嘅力量。



There is a sense, that Christianity, is fundamentally inconsistent. It is praxalogically inconsistent. It says one thing but does another - or rather - it says one thing but requires something else for it to survive and prosper. It is the same thing why “I vow to thee my country” can be argued to be fundamentally a warmongering, fascist, and therefore ultimately unchristian song. If Christ asks you to turn the other cheek when your crimea is taken, and your twin towers obliterated, and your country raped, then to vow to thee my country all earthly things above is unchristian - you ask for God to be on your right hand to make wealth and his wisdom and love on your left to forge the weapons so you may slaughter his children… yet Christianity like all other faiths - including science - which is a faith if you’re honest and you dive deep enough - cannot prosper if its prophets are artful in speech but ultimately unarmed… Christianity needs armed prophets but it also demands its soldiers be unarmed - it is blatantly inconsistent. 


Who is good enough to enter heaven? The theologically orthodox answer is no one. Everyone is a sinner. And no one is able to make up for their sins. No one is morally perfect for heaven. Entry to heaven is only by God’s grace. 

This seems to me to be saying that ultimate moral goodness is not the result of some supreme moral algorithm. It is ultimately determined by god - an intelligence beyond algorithmic compatibility. 



This outrageous theory by Julian Jaynes that I have first heard from a friend in New York earlier this year has inspired me an equally outrageous thought in me. Jaynes argue from the Illiad that before 1200 bce humans did not conceive of “the inner voice” as an inner voice of one’s self, but as a voice from god or from the gods. The breakdown of that conception which gave way to the recognition or identification of the inner voice marks the a milestone in the evolution of consciousness. 

One thing that has befuddled me is Christianity’s emphasis on love. It has always seemed so bizarre to me that Christianity should place such emphasis on love as the most sublime form of human emotion - in the sense that the position appears so very sounding loud the obvious. OBVIOUSLY love is the most profound and sublime affection humans are capable of duh. (係愛啊哈利) well, perhaps the reason why Christianity made such a big deal out of it is because humans WERENT capable of love. (Perhaps the psalms would disagree) Love appears to be apparently and obviously the most sublime and divine emotion there is only because Christianity has triumphed. Indeed, on an arguably ignorant perspective, can we say for sure the Romans knew love, with all their political marriages, warmaking, boy-fucking, and divorcing? Did the Greeks know love? Did the ancient Persians? The Egyptians? Did the ancient Chinese really express “love” in 青青子衿 悠悠我心 縱我不往 子寧不嗣音? Is 所謂伊人 在水一方 really an innocent expression of playful flirtiness across a stream, or is it a grotesque and lustful display of “no means yes”? 



“If and only if” propositions are engines of truth. They beget a paradigm of truth in themselves. They assert a web of logically and metaphysically connected assertions. If “existence must have meaning” implies and is implied by “god exists” then in the chain of proofs that establishes the implication and the contrainplication we will have established an associated chains of truths as well. They live or die with the conclusion. 

If existence has meaning is what we take to be truth on faith, in the same way we take the law of non contradiction, we may say that the

If we seek morality only because we are driven by a deep drive to continue existing, because we want to realise that meaning, we will ask where does that meaning come from, and god would be a reasonable answer. 


The theist answer to the Question of Existence, Morality, basically the question of what is the true the good the right beautiful, must not only sssert god to be an answer, it must assert to be the only answer 







