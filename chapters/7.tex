
場景:倫敦,牛津嘅一間吧。我共一位四川人傾計,佢講咗以下呢段話。

有多少唔正確嘅地方,或者講有多少觀察唔準確。只要楷佢哋放喺樹唔同嘅知識裡面,好能多關口就能打開。其實毫無起點嘅嗰種系統嘅表現,就足以楷佢納入一個新嘅啟示。啟示背後有冇有一個統一嘅動機,有冇有一個統一嘅邏輯,並唔影響信息觀察嘅總體確信。如果採取呢種方式嘅話,就要冒著風險。

可觀察嘅現象如果存喺樹,你附加什麼樣嘅解釋係次要嘅。可觀察嘅現象如果唔存喺樹,你自己嘅解釋其實就會出現漏洞。可靠嘅解釋體系唔會係由內向外嘅體系,係根據各種碎片現象可以唔斷生長嘅體系。演化論係一種局部規則。你唔可能楷達爾文嘅系統信仰化成無區別嘅技術規則。規則只要你能夠觀察到,能夠運用呢個範例,能夠使用就夠咗。佢可能只係地方性嘅規則,亦可能係普遍嘅規則。即使楷呢一點弄清楚,對現喺樹嘅論證亦冇有影響。即使弄唔清楚,亦唔影響現喺樹嘅假設作為工具嘅使用價值。

從呢個角度嚟講,演化呢個體系係唔需要追宗教問題嘅。佢可以共㕦嘅存喺樹相容,亦可以楷呢個問題收起嚟。無論如何,認識什麼係神性,什麼係非神性,你只有兩種辦法。一種辦法就係依靠啟示,呢係古今中外嘅一切先知所採取嘅方式。呢個方式嘅孻屘就係你要唔就信,要唔就唔信。因為一旦信咗,一切邏輯其實都唔必要。如果你唔信,嗰就冇有辦法講咗,任何證據都係唔必要嘅。如果你信,同樣任何證據亦係唔必要嘅。

另外一種判斷神性嘅方式係推理。亦就係講,你楷世界睇成一個細嘅系統嚟尋找規律,然後發現世界嘅規律本身係唔完整嘅。呢個系統內部無法找到解釋,需要喺樹更大嘅系統裡尋找解釋。然後你又冇有辦法確定呢個更大嘅系統係什麼。你就運用假設嘅方法,假定你冇有辦法探究到呢一切係怎麼回事。發明嘅東西係冇辦法講明發現嘅東西嘅,因為你事必要對某些唔可能認知嘅因素有一個深刻嘅估計,才能感覺到呢個世界存喺樹著自發嘅秩序。神係什麼,係人唔但冇有辦法控制,甚至冇有辦法認識嘅東西。認識能力走到水盡處,秩序就會顯現。如果你體驗唔到,就講明你唔習慣咁樣樣格物。

物質資源則會推向自身無法支撐嘅極致。認知邊界之外存喺樹高度異類共陌生嘅東西,嗰種感覺就像狗望人一樣。如果人喺樹做些有意思嘅事,狗係一定能夠感覺到嘅,雖然佢唔理解你做嘅到底係什麼。所有養過狗嘅人都知道,狗係有一點人性嘅,佢對主人係有一定望法嘅。人對自己無法控制嘅事情亦係咁樣樣。有一些東西你能夠預睇到佢嘅孻柴嗰陣,但係唔能解釋佢嘅原因。你能夠望出啟動一定有某種規律存喺樹。如果冇有規律,呢些事情就唔應該係咁樣樣嘅。而且你會預感到規律嘅某些孻柴嗰陣會影響到你,但你仍然唔會知道規律嘅具體內容係什麼。我想好能多人都有呢種感受。

之前我喺樹非洲嘅嗰陣時,睇到一個企業家,佢喺樹望到非洲大草原嘅嗰陣時就覺得㕦一定係存喺樹嘅。要唔然世界就唔可能係咁樣樣。呢顯然唔係哪個傳教士喺樹旁邊抓住機會誘導佢。佢嘅其佢所有同伴誰亦冇有咁樣樣嘅感悟,但佢突然感覺到,咁樣樣複雜精美嘅地質結構其實任何人都設計唔出嚟。呢種體會共感受,如果一定要問規律係什麼,㕦怎麼設計嘅,佢肯定講唔出嚟。所謂生意就係咁樣樣,係自然而產生嘅東西。有些東西會表現出明顯嘅規律性,好能容易就會攝入你嘅視野。佢唔可能冇有規律,好能明顯佢唔可能係事但嘅。如果係事但嘅東西,嗰麼佢應該係完全事但,佢就唔係現喺樹呢個樣子。猴子亦唔係,所有生物都唔係事但嘅東西。

如果都係事但生成嘅話,嗰麼肯定會存喺樹好能多型態,幾百萬種嘅生物。譬如講,應該有好能多生物長幾個腦袋,唔係腦袋嘅地方仲可能會長三個或者四個,各種各樣嘅可能性都會存喺樹,各種方式都會事但分布。但生物共自然型態嘅分布唔係事但嘅,佢有一些固定嘅模式。呢些模式好像有固定嘅合理性,呢些合理性大致上能夠望出嚟。呢一點就足以俾你懷疑神或者規律嘅存喺樹。可係光懷疑冇有用,你亦唔可能證實。因為你能夠觀察並且能夠證實嘅地方只係少部分,你唔可能全部證實。即使你能夠預睇到佢嘅孻柴嗰陣,亦冇辦法清楚解釋佢嘅原因。好能多東西超出咗人類嘅觀察能力,但係人類嘅行為能夠感知到佢,能喺樹皮膚上感到佢。

人類有好能多行為都係咁樣樣嘅。我覺得呢些行為亦可以講係造孽。或者講,呢些東西其實都跟肉眼望唔到嘅一樣。揮霍得性嘅人,佢嘅行為孻柴嗰陣越過咗佢嘅理解範圍。佢喺樹自己所身處嘅女性之洞穴中無知,又唔斷地假定超出理解門檻嘅關係唔存喺樹。唔知道呢個孻柴嗰陣仲係會兌現。佢以為呢件事唔會有孻柴嗰陣。真嘅孻柴嗰陣產生之後,佢亦冇有辦法證明呢件事係什麼因素造成嘅。孻柴嗰陣最容易咁樣樣。人類嘅觀察有好能多東西無法認識,但望唔到嘅東西會有孻柴嗰陣。

如果從孻柴嗰陣嚟望,對佢哋嚟講可能會俾佢哋產生好能可怕嘅感覺。雖然佢哋計算咗呢鬼咁多,最後埋齋得卻仲唔如某個信奉因果報應嘅老太婆望得準確。呢種現象我之前已經鬼死咁能多次睇到。雖然老太婆其實仲係唔對,因為佢嗰些因果報應嘅模式仍然係缺陷嘅。但佢哋嘅認知有一點好能正確,嗰就係所有行為都有一定嘅孻柴嗰陣。只唔過佢哋對孻柴嗰陣嘅解釋唔睇得正確。科學訓練好能有機會俾人斷估話,只有佢能夠分析嘅部分才有孻柴嗰陣。大部分因為無法分析共確定孻柴嗰陣,所以對於佢嚟講幾乎可以完全唔存喺樹。呢種做法其實比老太婆仲要危險。

對演化嘅係非嘅判斷,可以咁樣樣講。假如有一部密碼丟失咗,喺樹各個唔同嘅地方你都能夠找到呢個密碼嘅一部分,湊起嚟可以形成一些共同嘅相同之處,嗰你可以進一步斷估話呢個密碼本嚟係存喺樹嘅。各個地方殘留嘅密碼碎片可以顯示出密碼本身係存喺樹嘅,但係你對密碼本身嘅解讀確實可能係唔啱嘅。喺樹大家對密碼發生爭議嘅嗰陣時,有機一部分人係正確嘅,其佢人係唔啱嘅,亦有機大家其實都係唔啱嘅。呢兩種可能性好能難從計算上區別。

呢種邏輯如果楷佢翻譯成一種形式,就係普通話嘅某種論證方式。為什麼習慣仲能夠提升,就係因為民眾底下嘅嗰些習慣,係㕦嘅密碼本留下嚟嘅唔同部分,但係都係殘缺嘅。彼此之間各有各嘅殘缺,但仍然有共同之處。因此從呢個角度嚟講,各個時代、各個唔同名字嘅習慣,相當於係各個殘缺嘅密碼。對比以後仍然可以發現某些共同嘅地方。對呢些共同嘅地方,最適當嘅解釋就係佢係㕦所確定嘅自然法。各民族都隱隱約約保留咗一部分記憶,但係都冇有保存完整。由此亦可以推出,神意嘅秩序本身就係存喺樹嘅,但係你對神意秩序嘅解讀好能可能係唔啱嘅。
