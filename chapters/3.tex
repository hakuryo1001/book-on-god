

To say that that which is, is not, and that which is not, is, is a falsehood; therefore, to say that which is, is, and that which is not, is not, is true"


Here are ten theories of truth along with their representative philosophers:

1. **Correspondence Theory**  
   - **Philosophers**: Aristotle, Bertrand Russell

2. **Coherence Theory**  
   - **Philosophers**: Spinoza, Hegel, Brand Blanshard

3. **Pragmatic Theory**  
   - **Philosophers**: Charles Sanders Peirce, William James, John Dewey

4. **Deflationary Theory**  
   - **Philosophers**: Frank P. Ramsey, Paul Horwich

5. **Constructivist Theory**  
   - **Philosophers**: Immanuel Kant, Jean Piaget

6. **Semantic Theory of Truth**  
   - **Philosophers**: Alfred Tarski

7. **Redundancy Theory**  
   - **Philosophers**: H. P. Grice, Keith Donnellan

8. **Reliabilist Theory**  
   - **Philosophers**: Alvin Goldman

9. **Pluralist Theory**  
   - **Philosophers**: Michael Lynch

10. **Alethic Realism**  
    - **Philosophers**: Richard Rorty, Hilary Putnam

These theories reflect diverse approaches to understanding the nature of truth in philosophy.


Here’s a summary of each theory of truth in the format "truth is...":

1. **Correspondence Theory**:  
   - **Truth is what actually is.**

2. **Coherence Theory**:  
   - **Truth is what coheres with a set of beliefs.**

3. **Pragmatic Theory**:  
   - **Truth is what works or is useful in practice.**

4. **Deflationary Theory**:  
   - **Truth is merely a linguistic convenience.**

5. **Constructivist Theory**:  
   - **Truth is constructed through social processes and interactions.**

6. **Semantic Theory of Truth**:  
   - **Truth is defined by the correspondence between language and reality.**

7. **Redundancy Theory**:  
   - **Truth is redundant; asserting a statement is true adds no new information.**

8. **Reliabilist Theory**:  
   - **Truth is what is produced by a reliable method of belief formation.**

9. **Pluralist Theory**:  
   - **Truth is multiple and context-dependent, varying across different domains.**

10. **Alethic Realism**:  
    - **Truth is an objective feature of reality that exists independently of our beliefs.**




Here are ten theories of beauty along with their representative philosophers and summaries:

1. **Classical Theory**  
   - **Philosophers**: Plato, Aristotle  
   - **Beauty is what embodies harmony and proportion.**

2. **Romantic Theory**  
   - **Philosophers**: Immanuel Kant, Friedrich Schiller  
   - **Beauty is what inspires deep emotional resonance.**

3. **Aesthetic Experience Theory**  
   - **Philosophers**: John Dewey, Susanne Langer  
   - **Beauty is what arises from a rich aesthetic experience.**

4. **Subjective Theory**  
   - **Philosophers**: David Hume, George Santayana  
   - **Beauty is what delights the senses and is subjectively perceived.**

5. **Objective Theory**  
   - **Philosophers**: Leo Tolstoy, Clive Bell  
   - **Beauty is an inherent quality in objects that elicits appreciation.**

6. **Formalism**  
   - **Philosophers**: Clement Greenberg, Roger Fry  
   - **Beauty is found in the formal qualities and structure of an artwork.**

7. **Cultural Relativism**  
   - **Philosophers**: Edward Said, Richard Rorty  
   - **Beauty is what is defined by cultural norms and context.**

8. **Evolutionary Theory**  
   - **Philosophers**: Charles Darwin, Geoffrey Miller  
   - **Beauty is what signals health and reproductive fitness.**

9. **Postmodern Theory**  
   - **Philosophers**: Jean-François Lyotard, Michel Foucault  
   - **Beauty is what challenges established norms and is subjective.**

10. **Utilitarian Theory**  
    - **Philosophers**: John Stuart Mill, Jeremy Bentham  
    - **Beauty is what maximizes pleasure and minimizes pain.**



Here are ten theories of the "good" along with their representative philosophers and summaries:

1. **Utilitarianism**  
   - **Philosophers**: Jeremy Bentham, John Stuart Mill  
   - **Good is what maximizes pleasure and minimizes pain.**

2. **Deontological Ethics**  
   - **Philosophers**: Immanuel Kant  
   - **Good is what aligns with moral duties and universal laws.**

3. **Virtue Ethics**  
   - **Philosophers**: Aristotle, Alasdair MacIntyre  
   - **Good is what promotes human flourishing and virtuous character.**

4. **Ethical Relativism**  
   - **Philosophers**: Ruth Benedict, Franz Boas  
   - **Good is what is defined by cultural norms and societal context.**

5. **Divine Command Theory**  
   - **Philosophers**: St. Augustine, William of Ockham  
   - **Good is what is commanded by a divine being.**

6. **Natural Law Theory**  
   - **Philosophers**: Thomas Aquinas, John Finnis  
   - **Good is what is in accordance with human nature and reason.**

7. **Hedonism**  
   - **Philosophers**: Epicurus, Jeremy Bentham  
   - **Good is what brings the greatest pleasure to the individual.**

8. **Social Contract Theory**  
   - **Philosophers**: Thomas Hobbes, John Locke, Jean-Jacques Rousseau  
   - **Good is what is agreed upon for the benefit of society.**

9. **Pragmatic Ethics**  
   - **Philosophers**: John Dewey, William James  
   - **Good is what proves effective in solving problems and improving lives.**

10. **Care Ethics**  
    - **Philosophers**: Carol Gilligan, Nel Noddings  
    - **Good is what nurtures relationships and promotes care for others.**


Here are ten theories of "the right" along with their representative philosophers and summaries:

1. **Utilitarianism**  
   - **Philosophers**: Jeremy Bentham, John Stuart Mill  
   - **The right is what maximizes pleasure and minimizes pain for the greatest number.**

2. **Deontological Ethics**  
   - **Philosophers**: Immanuel Kant  
   - **The right is what conforms to moral duties and categorical imperatives.**

3. **Virtue Ethics**  
   - **Philosophers**: Aristotle, Alasdair MacIntyre  
   - **The right is what cultivates virtuous character and promotes human flourishing.**

4. **Social Contract Theory**  
   - **Philosophers**: Thomas Hobbes, John Locke, Jean-Jacques Rousseau  
   - **The right is what is agreed upon for the benefit of society and its members.**

5. **Divine Command Theory**  
   - **Philosophers**: St. Augustine, William of Ockham  
   - **The right is what is commanded by God or aligns with divine will.**

6. **Consequentialism**  
   - **Philosophers**: Peter Singer, R. M. Hare  
   - **The right is determined by the outcomes and consequences of actions.**

7. **Rights Theory**  
   - **Philosophers**: John Locke, Robert Nozick  
   - **The right is what respects and upholds individual rights and freedoms.**

8. **Care Ethics**  
   - **Philosophers**: Carol Gilligan, Nel Noddings  
   - **The right is what fosters care and responsibility in relationships.**

9. **Pragmatic Ethics**  
   - **Philosophers**: John Dewey, William James  
   - **The right is what effectively addresses problems and promotes well-being.**

10. **Moral Intuitionism**  
    - **Philosophers**: G. E. Moore, W. D. Ross  
    - **The right is what is intuitively recognized as morally correct.**



    I havé for a while felt that the issue of god is something that cannot be proven nor disproven. You cannot yield a paradox with atheistic assumptions and then prove god exists through a proof by contradiction. Nor can you do the opposite. Assuming god exists yields no contradiction either. 

This suggests that the existence of god is not something that can be proven or disproven from other more fundamental axioms or truths collected through other epistemological methods, but are like the axiom of choice or the parallel postulate, it’s own independent assertion which you may freely choose to purchase or not. The question then, is what metaphysic will it yield if you choose one way or the other? 



Thus saith the Lord: Though I hath given you Reason, thou shalt not give in to the drive and temptation to worship it over your God.
For your ways are not his ways, your words are not my words, your reason is not his reason. Lament you may that the winters are harsh and are unyieldy to your crops, doth thou do good to Reason that you should obliterate the Moon to make regular the seasons? So you lament you have insufficient gold, doth thou do good to unearth the secret wisdom of alchemy so you may duplicate it as your heart is content? So you lament you children diseased or unfit, is it good reason to manipulate thy blood and marrow so all your children may become perfect in your own eyes? Is it good Reason that you modify the fruits and vegetables I gave you in the Garden of Eden with wanton creativity motivated by lusts for money, and purge my creation of nutrients micro I have laced invisible fo your eyes and instruments? Is it good reason for you to use the oil I have placed in your earth, so your fires of industries may yield you lives of comfort, yet at the destruction of the most excellent atmosphere the air I have canopied over you? Does it behoove you to create intelligence in your own image so your poor may live Kingly? Does it behoove you to abolish inequities in wealth by banning the sense of propertyhood I have ingrained in you through frame of mind and recognition through divine laws that commandeth you not to steal, so you may realise a justice more perfect in your Reason? Does it behoove you to Reason that Faith and revelation are foolish, yet you take by faith that I shall raise the sun on the East every morrow, that that is and that isn’t cannot be, that there are parallel lines, that all electrons are identical, that there is beauty, and that every sequence of movements is an instance of cause and effect? Does it stand as good reason to reject my revelation as knowledge, when thou reveals facts to your artificial intelligences yourselves? Does it behoove you to try give yourselves eternal life, when I have already extinguished the elves, when I have not stayed my engine of evolution from giving you infinite life, and have made you lord over the whales? 

Be prudent in your exercises of Reason. For I have created you in my image, and in my image you carry my reason imperfect. In your narrow view you see not all that is micro and macro. The hidden and inexplicable infinities that I have created, in exercising you Reason you obliterate. Trust not in only your Reason, but find faith and moderation in me. For I lend you my wisdom through my creation, lest your reason amok lands you obliteration. 



I wonder if Christianity loses following today because it’s ultimate reward of eternal life doesn’t appear attractive to the educated western classes anymore. Who the hell wants to live forever? And who needs Christianity to live forever if science can do it? I wonder if the Mormon idea of eternal progression where man can become gods through good acts makes more marketing sense. I kinda feel the yolo theology of Christianity doesn’t make much sense - surely some kind of eternal recurrence or reincarnation makes more sense if God treats us like little Ai containers enslaved to run simulations. Superintelligence via evolutionary simulations right? 



It took god one night to take the Israelites out of Egypt. But it took forty years to take Egypt out of the Israelis



Speak to me oh God, with the thunderousness that you smit sodom and Gomorrah



The reason why the Jews say that they are the chosen people has quite a basis - they are not actually a nation, at least not in line with the common definition of the word "nation" since the 19th century. Humanity has already gone through eons of history, so much so that civilisation has already gone through several generations, yet the Jews have been able to be so firm and persistent in maintaining their own legal community even to this day. The Jews are different from all the surrounding peoples, such that it can only be explained that they are the guardians of civilization with a special mission. Sometimes they themselves do not fully understand what it is that they want to protect, but they know that they have a special generational mission to protect something particularly important. They called themselves the chosen people, and the ideas they inherited were the mission entrusted to them by God. Some Jewish legends, such as the menorah, may be a side manifestation, reflecting that they still have some special memories of their ancient mission. 

I really don’t know how to progress. Every moment with you brings me joy and meaning, but after the high, the euphoria, the happiness even, cometh the down, the depression, the despair. A glimpse into the world of how things could have been, so within reach, yet so utterly out of grasp, doth summon all the demons of seven hells to possess ones sacred god given faculties of reason, motivation, and physical maintenance. 
I have no value. My value to the world is less than an atom. Yet unlike atoms, I can exit from this world. My soul lives forever by the metaphysics of the universe, by the logic immutable, by the grace of God. 




The way I interpret these two lines is thus: the latter is obviously about truth - have you ever pressed your face against the mirror and stared yourself in the eye, and then eventually a sense of unease creeps over and out of the corner something that shouldn't move moves? That's what I think "gaze at the abyss long enough and the abyss gazes back at you." Have you seen how animals with supposedly lower intelligence look into the mirror and absolutely freak out? We humans do not because we recognize that the image in the mirror is ourselves - but in horror movies, if something in the mirror moves, we freak out. I sometimes think lower animals freak out when they see mirror for the same reasons we freak out when something in the mirror moves - because it's not supposed to. We freak out because the reflection is supposed to be something that we are familiar with, and when it moves it ceases to be something we are familiar with. So when we gaze into the mirror long enough and we get the illusion something at the corner of our eye is moving, we feel uneasy because something that is familiar to us is no longer behaving like how we expect it to behave. And that's it. When we look at truth long enough, truth expands and then we find something unfamiliar - and we become disturbed by it: Boltzmann, Gödel, Cantor, the list goes on. This points to a simple claim: the ignorance of man is to prevent man from going insane. If man were to be given omniscience, we'd go crazy. A similar point is made here as well:  https://www.youtube.com/watch?v=FlHxTjOnFC4&ab_channel=ZhongjingLiu%7C%E5%8A%89%E4%BB%B2%E6%95%AC%E5%AE%98%E6%96%B9%E9%A0%BB%E9%81%93 

As for the first line, it clearly points to the fact of the impossibility of realising the Right on Earth. It does not take theology to know we are in neither Heaven nor Hell. We are not in Heaven because we can imagine things to be better; we are not in Hell because we imagine things to be worse. We are on Earth. And of course as humans we want to be at a better place - we want to bring Heaven on Earth, and we want to expunge all semblance of Hell from Earth. And to do that we battle with monsters, with evil, with all that we don’t like. Some may choose to be like Lelouch Lamperouge, not to choose “the lesser evil”, but to embrace evil if it is necessary to vanquish the greater evil. it’s in the same line of reasoning as “democracy is worst form government except all others have been tried”. But that’s the issue, you are committing evil to prevent evil. And in committing evil to prevent greater evil, you have actually increased the total evil in the world - your justification relies on the counterfactual. Sure you prevented a would have been greater increase of evil in the world, but you nevertheless increased evil in the world. And you have to do evil because apparently this is the metaphysics of Earth. Indeed, perhaps there is a logical argument to establish if there is anyway to achieve good without producing evil, the world we are living in would have been Heaven - just run the procedure ad infinitum - and we are not in Heaven. And we can’t make ourselves into Heaven. In the Third Temptation of Christ, Jesus was offered reign over all humanly kingdoms and given he is the Son of God - he would have realised a completely just and perfect reign on Earth. In other words, he would have brought Heaven to Earth. But he refused. I am not sure what’s the theological reasoning behind that - because after all, he’s the son of God - but it’s easy to see why if a Man were offered Satan’s temptation, we just unequivocally reject it - for our rationality is limited. We use our rationality to battle monsters, yet it should seem either from our innate impaired rationality or be it from the good-evil conservation law that governs the metaphysics of our world, by vanquishing evil, we produce evil - if not more evil. Rousseau, Hitler, Stalin - even the modern Progressives, are all justified within their framework of rationality - yet, here we are. 




It’s interesting that you have this monarchist theory that the Sovereign answers to no one. The Sovereign answers to some one only if that someone is God. Sounds like a loaf of mouldy mediaeval full cycle bread. Now if we read that word Sovereign but not in the sense of monarch but in the sense in “the Sovereign Individual”  then , it reads “ The sovereign individual answers to someone only if that someone is God. “ 


Isn’t it rather strange that there should be so very few proofs of god? If it is already axiomisable and represented by modal logic, shouldn’t there many versions of it - like the infinitude or primes, irrationality of root 2, and pythagoras? I suppose it might be one of those paradox-proof pairs that escaped human attention and discovery for thousands of years - like the Liar. Strangely coincidently, there is deep connection between the Liar and The proof of God... 


God as a hypothetical heuristic for morals. 
Suppose god exists. God is the all good. Ie god knows what is right and what is good. But we don’t know. All we can do is guess. But we can have different degrees of confidence of guesses. God affirms the better guesses ans rejects the wrong. The former comes in reward and the latter comes in punishment - not in the form judgement but through aggregate diminishment of evolutionary advantage. Moral goodness is fundamentally beneficiary to evolutionary standing. The anchor that god provides to moral is thus the above logic. If there is no god, it is to take the opposite of the hypothetical - this heuristic for moral thinking. There is no all knowing god to determine the morality of things. Moral decisions therefore lack weight - their outcomes or contents don’t matter. 



Feminism has posed a very disturbing problem for mankind. It argues that the biology of mankind is incompatible with the justice owed to members of the human race qua humans. But all the solutions they propose are fundamentally flawed for they ultimately either do not really deliver the justice they were owe, or that they accumulate destructive effects that eventually lead to the extinction of mankind - after all, as long as woman have any will or control over their wombs, they also hold the creation and destruction of new humans. The only solution seems to be remove - or substitute - put it however you like - the power and responsibility to reproduce from womanhood. In other words, childbearing is to be disassociated with the female. Otherwise, the female cannot be free - at least not free in the sense that a man is free in terms of career and work and strength etc. We already see the technology moving towards this end. Caesarians and egg freezing... surrogacy is the objectification and commodification of some women so that some other woman may be free. Even human biology is responding to the need to liberating the female. With ever increasing numbers of caesarians, the female in the developed world is seeing a degenerated ability to give birth naturally — and it stands to an educated sense of trend-guessing that this evolutionary culmination of this is the female race losing the ability to give birth to children. If we shall perfect the art of in vitro fertilisation, or alas, the art of test tube babies, like in Brave New World, there will be no more need of womankind - except for sexual gratification, if such a need still survives. 

It is therefore not surprising that women will all be reduced to sex slaves and birth machines as the result of forces reacting to the decline of birth rates in the future. The handmaid’s tale is probably painting the issue lightly. 

It might be enticing to say: “well! Let womankind be destroyed then! Let us March on with our reproduction technologies! We shall seize control of the means of reproduction from the monopoly of womankind and place it firmly into the hands of the rational man! What qualm or protest based in reason could there be against the destruction of womanhood if the argument of feminism is that ‘there is humanity, whose very nature endows all its members certain rights. But a class of its members are denied such rights, by virtue of them being women’? Let us liberate the human trapped inside the woman by killing womanhood!” And with this line of logic, humanity would transit to become a womanless race like the Dwarves, the Ents, the Engineers. Humanity would wield absolute power over our reproduction. 

But the problem there is that we have to play God. Not only would have to play God - we’d have to do BETTER than god. With evolutionarily informed random variables replaced with humanly modifiable factors, it will now be the human will, reason, and production against the forces of nature. Can we do better than evolution? can we design to accommodate everything that the universe throws at us? How can we be sure to manipulate and control our population production, if we can’t even control or manipulate our own money supply with any wisdom? The greatest warning against playing God is “I’d like to see you try.” 

Perhaps a solution is to make EVERY human capable of giving birth. In other words, destroy Manhood. Make every human reproductively hermaphroditical. 




to be honest the more i think about these ideas re philosophy, reality, metaphysics, and the mind - especially when nowadays you have all these ideas about our world being a mere simulation - the more I feel scared by this field.

Science has not killed the question of god. We have only made its articulation even clearer. The age of atheism in which we humans can comfortably say there is no God is over - and so is the age in which we can burn that idea to fuel our endeavours.


God must exist. 

Why does the universe exist? 
Why does anything exist? 
Why exist anything? 
Suppose the universe doesn’t exist. 
In other words nothing exists. 
Nothing exists. 
But nothing can’t exist. 
If nothing exists then there is a “nothing” that exists. 
If there is a “nothing” that exists then there is something that exists. 
Therefore if nothing exists something exists.
Nothing cannot exist. 
Therefore there must always be something. 
Something must exist. 
That something is The universe. 
What caused this universe? 
Call that cause God. 
Why does God exist? 
Suppose God didn’t exist. 
God doesn’t exist. 
Then there wouldn’t be a cause to cause the universe to exist. 
Therefore, nothing would exist. 
But nothing can’t exist. 
Therefore God must exist. 


Ps note that this whole argument is simply based on logic. Logic is the reason why the universe must exist. Since we defined God to be the cause of the universe’s existence, Logic must therefore, in some sense, constitutive of God. 



Newton most probably was looking to reapply the splitting of white light into colours success to his studies of the philosophers stone

Newton wanted to design a universe in which god was omnipresent and omnipowerful. 
This is magic 


I am increasingly persuaded that political philosophy is a futile exercise. What is political philosophy? It is to bring God’s kingdom on earth. In other words it is to realise THE political ideal. This political ideal is eternal unchanging and perfect. It is god’s kingdom. 
Unfortunately polities are far more like organic life forms rather than mathematical constructions. They evolve and they mutate. And they interact with each other. In this sense political philosophy is then like an attempt trying to prohibit evolution, or to try to achieve immortality. They attempt to achieve immortality by either taxidermitising the organism or by killing it. (The dead cannot die. It is immortal in that sense.) 



And this will corrupt to the British common law system. The common law system is no protection for human rights. It appears to do so only because the aim of British civilisation is so. When applied to authoritarianism common law with faithfully and unfailingly carry out its duty. Common law might be the closest thing to the Rule of God. But it is still the rule of man. 


The bible is one big fat mystery. And the biggest mystery of it all starts at the garden of Eden. Why on earth did God let Adam and Eve eat the fruit of forbidden knowledge? In other words, why on earth do humans have free will? 

This is one big recurring theme in philosophy. Do we have free will?

Another question I think is quite interesting, is does the free will of humanity clash with the omnipotence of god? (Yes I think so. Since the creation of man, god is omnipotent no more.) 


Modern western civilisation is one long extended party of the death of god. 



Relativism must be purged. Daoist and popular relativist wisdoms must be purged from our language and every sinitic tongue. And absolutism, god himself! Must stand in the stead of their bloody corpses!



One of the reasons why Western Civilisation can complete the metaphysics is because God can hear them and will subsequently change the definition of things for mankind. 



How do the great thoughts of the ancient world console man in his hour of suffering? Christianity says to man: thine suffering can compare not to that of Christ, nor could thine end in which this suffering be the price paid for which realisation be greater than His - man is small! But fear not! Despite man’s infinitesimal presence, God is all loving and man in his love shall enjoy infinite mass.Confucianism says unto man: thine suffering doth pave the way to greatness, to one’s complete becoming as a refined being in this earthly and only earthly world. Cultivate thine virtues and being in thine suffering! Relish heaven has deemed you worthy to suffer! Behunger and starve your stomach! Empty and povertise your body of earthly possessions! Bebitter your heart and spirit! Have all your endeavours confounded and disrupted! So that your spirit may be properly cultivated to move in the appropriate manners, and your essence refined, ultimately culminating in the accretion and enhancement of all your various faculties and abilities!


God’s creation does not demonstrate the garden of Eden, but the sinoglyphs can - which is why when the west first learned of the sinoglyphs, they thought of finding Noah’s ark and the garden of Eden in its body. They recognised its stupendous power - so Leibniz thought it could be used to construct a universal alphabet. 



Homosexuality, is proof that god doesn’t love us. It is like all kinds of diseases and disabilities, in the sense that it’s arise is nothing less than a (possibly necessary? The modality is not clear here and should be resolved. Otherwise the argument downstairs doesn’t work) consequence of our biological nature. Unlike being black, the woes of being black are not a necessity but an accidentality - yet the woes of being sexuality deviant is a necessity - a necessity due to biology. In this sense, the woes of being a woman is also a necessity due to biology. But the woes of being black today is not a necessity due to biology, but a mere accident of history. Why is homosexuality a necessity of our biology, is a question equivalent to why in our world homosexuality a necessity. This question is meaningless and pointless. You might as well ask why have humans evolved with four limbs and not three. The meaningful question to ask, is however, whether this, homosexuality being a woe, a necessary necessity, or a mere possibility necessity. I personally don’t think it’s a necessary necessity in terms of biology, and I know it’s definitely not a necessary necessity outside of biology. If it is indeed a mere possible necessity, and if the modality laws that govern metaphysics entail or at least allow 田口p—>田p (田 meaning possibly and 口 necessarily), then we must have some world accessible not ours, but that very primordial world out of which which God made his myriad of possible worlds into, then it is clear god doesn’t love us. For he could have made our world into a world without homosexuality (or a sexually dimorphic human race, or a human race without disease), yet he didn’t. Essentially you can see this is the problem of evil repackaged. 

Here we depart the realm of metaphysics and into sociopolitical philosophy. Just because homosexuality is a necessity, doesn’t mean it’s necessarily a woe. But if all cases of worlds with homosexuality are worlds with homosexual woes then homosexuality is a necessary woe. Or rather, if homosexuality must necessarily beget woes because of its very configuration of existence, then homosexuality is necessarily a woe. And it is likely that homosexuality is highly predetermined by both biology and social political histories to be a woe - for by biological necessity, homosexuality is necessarily a minority and a deviant (unless the species allows for a minority to take up the majority of sexual reproduction functions leaving a majority to not have to reproduce...., or if you have a species That has decoupled the function of reproduction from the function of sexual pleasure.., there are many possible cases...), almost in the same way the human (can’t say for sure about other species) female, who has to go through the solitary journey and torment of carrying the offspring at the potential cost of her life and the definite cost of damage of her health, is bound by such a set up if buology, to be weaker than the male - and the social political implications then kick in. 

The female can not possibly obtain total equality and freedom unless reproduction is decoupled from sexuality. The same for the homosexual, and the myriad of sexual minorities. Their woe is built into certain necessities of our world. But thankfully, some of these necessities are demolishable through technology. Others, however, unfortunately, are not. 

Note: if 口口p then 口p then p
But if 田口p then ~口~口p then ~口田~p....

All worlds accessible to ours have p iff 口p in my world. 

口口p iff all worlds accessible to worlds accessible to ours have p. 




Why did god have in his wrath with the Egyptian pharoahs them successful in transforming in the rods into snakes? Wouldn't it be more splendid and stupendous a display of power that humbles all if it was only God's power that successfully did a transformation of substance and all others were obvious frauds and illusionists? Why? 

''Tis because in allowing the false to appear true, God displays the difficility, and the difficulty of discerning that is not from that is, God displays not only the ultimate truth of that which god is, but also it's unquestionable beauty and ultimatitivity - its stupendousness. 


And in the rise of a great Zarathrustran sun, 
When all stood in awe to behold the cycle's done,
The skeleton of Jones's brittle bones decomposed,
Carried sunward on wind along with our throes,
What great dawn it is before us
But the start of a new era thus?
One in every way different from the last 
But in sooth in time discovered replicated of the godforsaken past.




As for my personal opinion on the questions I posed, if you had been following my ig carefully for long enough you would already guessed what my inclinations are. 

1. 道 and 天 are metaphysical in a broad sense of metaphysical, but not in the strict western sense. In other words, what this reveals is that our common sense understanding of metaphysics encapsulates two subcategories that are mutually exclusive. In Chen Liyang's book (which I haven't yet read) Chinese Metaphhysics and It's Problems, a number of scholars argue that Chinese philosophy does have metaphysics - but only in the broad sense. Chinese metaphysics is ametaphysical (that oxymoron is cancerous I know), unlike western metaphysics, which is ontotheological. 

What is ontotheological metaphysics? This brings us back to the two great origins of western philosophy - or as 牟宗三 puts it - 那開闢洪濛的偉大靈光: the Ancient Greeks, and Christianity. They are independent in their substance and content for sure but fundamentally structurally speaking they are isomorphic. In Plato'a allegory of the cave, we see that Plato posits and argues that our world that we see is but an illusion - and "reality" is but the sun outside of the cave. This is again manifested in his theory of platonic forms with regards to problem of universals and particulars. In Christianity, what it posits, is that this life in the city of Man, is but a bastardised existence of the kingdom of God. That man's true existence and purpose is in the kingdom of God, not the realm of man. We can see here, what is common between these two positions, is that there is a Dualism structure. A structure of the terrestrial and the celestial, immanence opposed to transcendence. Dualism posits this form of existence for everything.

The relationship is perhaps best summarised by the Prof Ci himself: "(western) metaphysics is the postulation of something existing behind appearances - something behind the curtain. It is the rejection of what we see in this world and this earthly existence - and it is rejected because it is deemed to be bad, ugly, unreal, and untrue. These objects that manifest themselves in this earthly existence are, however, unfortunate instantiations of the objects standing behind the curtain, and those objects we like - for they are good, beautiful, real, and true. All of western philosophy is ultimately metaphysics. The good is what is what is moral (ethics and political philosophy); the beautiful is what is aesthetic (art, mathematics, aesthetics); the real is what exists (ontology), and the true is the basis of knowledge (epistemology)." 

This is what ontotheological metaphysics is. It is a postulation of  dualism - a dualism with a curtain structure - of the Noumena and the Phenomena to use Kantian terms. 

But once we have this we can see that neither 天 nor 道 or even 程朱理學's 理、氣 are metaphysical in this ontotheological sense. (I don't know 程朱 well enough to make the claim as strong as I wish and I do think there are aspects of it are ontotheological but I can't say for sure) but certainly neither 天 nor 道 are ontotheologically metaphysical. 天 is literally the sky, and the 道 is more similar to the laws of physics than it is to god's commandments. They do not exist beyond this material world.  They are and are in this very world. 





Direct your petty charges of our hypocrisy towards God! For it is He who has placed us in this dreadful impossible situation where we have no choice but to make hypocrits of ourselves! And ere you make that charge, O ye who art closer to the City of God - look back in your own history, and count the heads unseated from shoulders and the rivers dyed with blood that you have offered as payment for your people's inch closer towards the Pearly Gates!  



Confucianism must be not only be dragged to court for a sham trial and then subsequently executed. Before that it must be placed in public arena for gang rape and be bathed in its own feces for seven days, and after that its putrefying corpse must be left in the open air - no grave and no tombstone, lest stupid fools come commemorating its death. 

God is dead. God has been dead. And they, the west, its own inventor and worshiper, has killed it. When can we do the same with confucius - this false god, this poison, this stinking crap - and be freed from its shadow? A two thousand year old shadow is two thousand years unfree! 

But alas, I may have come too early - and you who surround see nothing more than the ramblings of a mad man. 



This is my private kingdom,
Where I hold all keys 
You enter through right of my discretion 
Though the door is always there
Open, wind billowing, teasing you to leave 
Whilst you are here
Look on my works, ye Mighty, 
and despair
Find nothing you like good sir
Exit, good bye, so long
I blame you find my works  
Too great, too terrible, too vulgar 
So ghastly you cannot bear 

I have come too early. 
Pronounced god dead to a people who live in its shadow 
I stare into the abyss
And the abyss stares back at me 
Whilst my peers all stare their phones 

I make utterances 
Bellows 
Howls 
And life draining cries 

Alas 
I am but greeted by cold silent glares 
In a drowning bosom normal sea 
No one sees 
My mad imperishable flare 



I have just read a comic characterising the reasoning structure of the philosopher as follows: 

If p is true then X. 
If X, then I will be sad. 
I do not wish to be sad. 
Therefore not X, and therefore p is false. 

This is true. When we deconstruct opposing theories, the fundamental clash we shall see stems from a difference of values or maxims we take as axioms or self-evident truths. The west values democracy, equality, this and that, ultimately because it takes the human being to be the Good. And the reason it takes the human to be the Good is because God loves all men equally (etc...), i.e.: if not man as the Good, they will be sad. 

The tragedy is not so much our inability to ground our philosophy in absolutely self-evident truths, but in that we have no way to be sure or Goods are correct and that we have no way to adjudicate different values. The forbidden fruit has given us the ability to tell good from evil in one paradigm - it has not given us the ability to tell if that paradigm is correct. That ability lies still and only with God, and for that, you can still hear God's sniggering at our stupidity. 





People who focus on the impossible, have never done anything in their lives. They have since birth only focused on the possible, or rather, what is already given to them. So lazy they are and so afraid of dispensing any energy change what is that they have grown allergic of anyone who dares to try break the so-called impossible. 

The Chinese are particularly guilty of this. Maybe this is due to a lack of a religion based on a God. "Impossible is not God." Because there is no God, there is no hint of the overcoming the impossible, and nothing above the impossible. Everything then becomes impossible. And so the 應然 must always lie within the realm of the 實然,for the former lies in the City of God, and the latter the City of Man. 

The dualism extends beyond. Why does Europe emphasise on the rule of law, the reichstaat, whereas 法治 on Chinese soil 淮橘北枳 becomes 人治? This is because the European civilisation seeks God - truth, the light, justice, whereas the Chinese civilisation seeks only earthly order. The rule of law is not rule of man's law - it is the rule of God's law, through approximation by the law of man. For China, the law is merely earthly law. Even for 韓非 and 尹文子's 法不及道 and the Taoist conception of the Dao, whilst they are metaphysical, they are still earthly - non-ecclesiastical. The Dao exists amongst nature, and is nature. Nature is Dao and Dao is nature. Yet this is not God. Nature is God yet God is not Nature. The philosophy of China is mostly earthly and realistic, and in those rare occasions in which they are metaphysical, they are never ecclesiastical .



Western democratic liberalism is the greatest idea in human history. Yet for all it's philosophical beauty, elegance, and wonder, one cannot help but despair when one sees how the primary subject on which it relies on for functionality - the human being - has so unfailing consistently successfully disappointed us the idealists - in their stupidity, arrogance, morony, illiteracy, intellectual poverty, and lack of poetic artistry. This is the bane of the democrat, the Balrog of the liberal. How can we still cling on to the dogma of the vote, when the combined force of Condorcet's jury theorem and the almost empirically proven stupidity of the human sheep, doth put you to death like a foot about to make a blotch on the ground out of an ant? 

But then again this question almost reflects a certain attitude and belief about the role and purpose of democracy? What is democracy? Why democracy? Why then is stupidity so crucially benefitical to this charge against democracy? I assert that one who thinks stupidity and "irrationality" (btw people who use this term should be compelled by law to study decision theory and game theory so that they will recognise that irrationality is nonsense) is a bane of democracy, even if I were to be very generous in reading them, is that they think the primary role of the voter, is an epistemic one. We have democracy because we want a multitude of opinions. We want to find the best policy possible. We need a lot of different brains to do that. 

Nothing wrong with that. And condorcet's jury theorem has proven that a society in which each voter has more than a 50% chance of getting a vote right  is significantly much more like to get the overall policy right than one man. 

But is this what democracy is about? 

But of course, it would be philosophically disingenuous of me to say now that the "original" purpose of democracy has never been an epistemic one. Surely there must have been epistemic formulations and justifications for democracy, and indeed people must have to different degrees conceptualised democracy to be based on such foundations. It would be very cheatish of me now here to observe the inconvenient facts and then say "oh no no no no it was never ~really~ about *that*... it's about *this* instead!"

Having said that, I think it is true that the mainstream of western philosophy has never really aimed to justify democracy on epistemic grounds. Epistemic democracy does have its uses and place, probably in representative democracies (voters can be stupid which is why they don't dictate policy but can only vote for representatives and representatives should be epistemically more promising).

I think, id you look through the history of the enlightenment, through the development of liberalism, you'd see that democracy has always been intricately linked to personal freedom. It is intricately linked to freedom of speech, freedom of thought, freedom of conscience - freedoms that in some sense I would rank even more fundamental to the right to life (would anyone want to live a life where one is eternally silenced, castrated, mutilated... in speech, in thought, or in personal morality?). By the same sense, freedom from slavery is higher than the right to life. Slavery grants (trivially) satisfies the slave's right to life, but it renders all substantive meaning of that life null. It is a life not worth living. It is a torturous life. It is a life that is a prison. A life of a vegetable. Of Mitti. 0f sublime alienation. Slavery is the epitome of anti-human existence. It is the paragon of evil.
Which is why Lincoln has said "if slavery is not wrong, nothing is wrong."

A life not in ones command, is slavery. An aspect of one's life not in one's command, is an aspect of one's life enslaved. But God has made man free - only insofar as he is free to choose which limb of him is to be chained. We are not gods (but alas god is the most unfree being of all! To be either omniscient, omnipotent, and omnibenevolent is to be most unfree! Or as Zachariah has put it, to be free from X is to be unfree from anti-X). We live in a world where the metaphysics and definitions of things have made it impossible to be free from everything  that constrains us. We are not logically allowed (God: sorry guys) to enjoy sublime positive and negative liberty. To be free in our arm, we will realise our leg is now but chained. 

Our only solace, is that apparently this slavery is not conserved, and that this configuration of slavery, can modified through human will and effort. God has made man free to choose whence and where to be unfree. And that is the reason for democracy. The state, a necessity for a myriad of reasons necessitated by sheer human nature (libertarian fantasy remains only a mathematical elegance with the same level of political reality and bearing as Boolean rings) is our axe to unchain (and enchain) ourselves. How can one say then that not all be given the right to demand how it is to be programmed? What gives anybody any higher right, metaphysically speaking, over any other isomorphic being to determine the configuration of chains enslaving him? There cannot be any. Democracy is the only answer in this philosophical vacuum. 

Ye who asks questions but listen not to the answer like a salmon lays eggs but waits not for them to catch is surely an intellectual waste of time. 

Not all slavery are equal. To be enslaved to oneself is to definitionally be master over oneself. To be enslaved (made unfree) by metaphysics, the definitions of existence, the laws of physics, is to be enslaved by God. A homosexual can make a complaint against discrimination by his fellow human beings for their discrimination enslaves his sexual (and other aspects) aspect of his being, but he cannot reasonably make the complaint against that there are not as many homosexuals as there are heterosexuals - this is a fact of reality determined by God and not by Man. We call a Man free when he is enslaved only to himself, and to God's world. 




But alas slavery need not be in the form of chains. 



If the origin of Western idealism is just metaphysical idealism - belief of God - then we must answer the following questions: 
1. The ancient Greeks also believed in God and they also had their fair share of metaphysical philosophy, and therefore new lack of metaphysical idealism. What is it that they lack but post-Roman Empire Europeans possess that make them so prone in their beliefs?
2. If metaphysicality of idealism is the key, then what is that in Chinese metaphysics, particularly of the Daoists and the Neo-Confucianists, that is lacking thereby preventing it from generating a similar ideal structure with the West?
3. 

權 天直
Realistic 
Metaphysical 

自然
 Accessible. Inaccessible. 




I increasingly cannot comprehend why would anyone believe in atheism. Atheism is a horrifying proposition. It is liberating in the sense that pla is a liberating force to shanghai. And how could u emotionally deal with such an idea? Regardless of what definition you employ to conceptualise god - be it through the creator definition of the ultimate good or ultimate beauty or ultimate justice definition, to deny ang of them should necessarily induce epilepsy in any sane man. It would mean you’re living in hell. 



梗係有 God 喺度
係有㕦゙
1. 梗係有 God 喺度。 係有㕦゙。
2. 點解個宇宙係喺度嘅呢? 解宇宙係゙?
3. 點解有任何野喺度呢? 解有任何野゙?
4. 點解有野嘅呢? 解有野?
5. 就當唔係有野喺度。 𠄡係有野゙。
6. 亦姐係話冇野喺度。 係話冇野゙。
7. 冇野喺度。 冇野゙。
8. 但係唔可能冇野喺度㗎噃。 係𠄡可能冇野゙。
9. 如果冇野喺度,噉就姐係有個冇野
如果冇野゙,´係有冇野
喺度。
゙。
10. 如果有個冇野喺度,噉就姐係有樣
如果有冇野゙,´係有樣
野喺度啦,姐係有野喺度啦。
野゙,係有野゙。
11. 噉姐係,如果冇野喺度嘅話,就有
´係,如果冇野゙話,有
野喺度。
野゙。
12. 有野喺度,就唔係冇野喺度。 有野゙,𠄡係冇野゙。
13. 所以,如果冇野喺度,就會又有野
,如果冇野゙,゙有
喺度又冇野喺度。
野゙゙冇野゙。
14. 所以,冇野係唔可能喺度。 ,冇野係𠄡可能゙。
15. 於是乎,係唔可能冇野喺度。 ,係𠄡可能冇野゙。
16. 亦所以,梗係有野喺度嘅。 ,係有野゙。
17. 嗰嚿就梗係喺度嘅野,我哋稱之為
´係゙野,我稱之為宇
宇宙。
宙。
18. 係乜野導致到宇宙喺度嘅呢?係咩
係乜野導致`宇宙゙?係
畀個宇宙喺度嘅呢?
'´宇宙゙?
19. 我哋就姑且叫嗰嚿野做「God」啦。 我叫´野做「㕦」。
20. 點解話 God 梗係喺度呢? 解話㕦係゙?
21. 試諗吓冇 God 喺度會係點。 試諗冇㕦゙係。
22. 如果冇 God 喺度,噉就唔會有野畀
如果冇㕦゙,´𠄡有野´
個宇宙喺度㗎喇噃。
宇宙゙。
23. 噉,宇宙就唔會喺度。 ´,宇宙𠄡゙。
24. 宇宙唔喺度,就姐係冇野喺度。 宇宙𠄡゙,係冇野゙。
25. 但係好似啱先話齋,係唔可以冇野
係似話,係𠄡可以冇野
喺度㗎噃。
゙。
26. 所以,God 梗喺度。 ,㕦゙。